% Options for packages loaded elsewhere
\PassOptionsToPackage{unicode}{hyperref}
\PassOptionsToPackage{hyphens}{url}
%
\documentclass[
]{article}
\usepackage{amsmath,amssymb}
\usepackage{iftex}
\ifPDFTeX
  \usepackage[T1]{fontenc}
  \usepackage[utf8]{inputenc}
  \usepackage{textcomp} % provide euro and other symbols
\else % if luatex or xetex
  \usepackage{unicode-math} % this also loads fontspec
  \defaultfontfeatures{Scale=MatchLowercase}
  \defaultfontfeatures[\rmfamily]{Ligatures=TeX,Scale=1}
\fi
\usepackage{lmodern}
\ifPDFTeX\else
  % xetex/luatex font selection
\fi
% Use upquote if available, for straight quotes in verbatim environments
\IfFileExists{upquote.sty}{\usepackage{upquote}}{}
\IfFileExists{microtype.sty}{% use microtype if available
  \usepackage[]{microtype}
  \UseMicrotypeSet[protrusion]{basicmath} % disable protrusion for tt fonts
}{}
\makeatletter
\@ifundefined{KOMAClassName}{% if non-KOMA class
  \IfFileExists{parskip.sty}{%
    \usepackage{parskip}
  }{% else
    \setlength{\parindent}{0pt}
    \setlength{\parskip}{6pt plus 2pt minus 1pt}}
}{% if KOMA class
  \KOMAoptions{parskip=half}}
\makeatother
\usepackage{xcolor}
\usepackage[margin=1in]{geometry}
\usepackage{graphicx}
\makeatletter
\def\maxwidth{\ifdim\Gin@nat@width>\linewidth\linewidth\else\Gin@nat@width\fi}
\def\maxheight{\ifdim\Gin@nat@height>\textheight\textheight\else\Gin@nat@height\fi}
\makeatother
% Scale images if necessary, so that they will not overflow the page
% margins by default, and it is still possible to overwrite the defaults
% using explicit options in \includegraphics[width, height, ...]{}
\setkeys{Gin}{width=\maxwidth,height=\maxheight,keepaspectratio}
% Set default figure placement to htbp
\makeatletter
\def\fps@figure{htbp}
\makeatother
\setlength{\emergencystretch}{3em} % prevent overfull lines
\providecommand{\tightlist}{%
  \setlength{\itemsep}{0pt}\setlength{\parskip}{0pt}}
\setcounter{secnumdepth}{-\maxdimen} % remove section numbering
\usepackage{fancyhdr}
\pagestyle{fancy}
\fancyhead[L]{STAT 550}
\fancyhead[R]{Consultation Report}
\fancyfoot[C]{\thepage}
\renewcommand{\headrulewidth}{2pt}
\renewcommand{\footrulewidth}{1pt}
\setmainfont{Times New Roman}
\usepackage{setspace}
\setstretch{1.5}
\usepackage{xcolor}
\usepackage{setspace}\singlespacing
\ifLuaTeX
  \usepackage{selnolig}  % disable illegal ligatures
\fi
\IfFileExists{bookmark.sty}{\usepackage{bookmark}}{\usepackage{hyperref}}
\IfFileExists{xurl.sty}{\usepackage{xurl}}{} % add URL line breaks if available
\urlstyle{same}
\hypersetup{
  pdftitle={Client Report Zero-calorie Drink Consumption Intervention},
  pdfauthor={Parham Pishrobat, Sarah Masri, Johnson Chen},
  hidelinks,
  pdfcreator={LaTeX via pandoc}}

\title{Client Report Zero-calorie Drink Consumption Intervention}
\author{Parham Pishrobat, Sarah Masri, Johnson Chen}
\date{2024-03-01}

\begin{document}
\maketitle

\hypertarget{introduction}{%
\subsection{1. Introduction}\label{introduction}}

Concerns around sugar consumption and its health implications have
prompted many interventions to shift consumer behaviour away from sugary
beverages. The current study investigates the effectiveness of various
strategies to encourage consumers to choose zero-calorie beverages over
sugary alternatives. In particular, the research question focuses on the
impact of two strategies to inform consumers about calorie content
through visual presentations: posters highlighting either the calorie
content or the physical activity required to burn calories. Furthermore,
the effectiveness of price discounts on behaviour is explored, both
independently and in conjunction with explanatory messaging.

\hypertarget{data-description-and-summaries}{%
\subsection{2. Data Description and
Summaries}\label{data-description-and-summaries}}

The primary outcome of interest is daily sales of bottled sugary and
zero-calorie beverages, starting October 27th for 30 weeks. The dataset
contains 631 sales counts for zero-calorie, sugary, and all drinks, sale
location, day of the week, and the type of intervention applied. The
study period includes a baseline data collection phase, intervention
phases to elicit behaviour change, and washout periods to assess the
persistence of intervention effects. The recorded variables consist of
sales count, day of the week, site location, type of intervention, and
beverage category (zero-calorie and sugary options). The price
interventions consist of two periods of 10\% discount on zero-calorie
beverages, with one phase providing additional explanatory messaging
about the discount. The calorie messaging interventions provided
information on the caloric content of sugary drinks, the physical
activity required to burn off these calories, and a combination of both
strategies.

The study adopts an interrupted time-series multi-site
quasi-experimental design to assess the outcomes of the five distinct
interventions on the purchase patterns of bottled sugary and
zero-calorie beverages. The data is recorded from cafeterias and
convenience shops within three hospital sites, denoted by A, B, and C.
Hospital A is urban and has two cafeterias and two convenience shops.
Hospital B is also an urban setting but has only one cafeteria. Finally,
hospital C is a suburban setting, having one cafeteria and one
convenience shop. Both interventions and data collection were automatic
at site A and by trained personnel in sites B and C. Sugary beverages
include regular soft drinks and iced teas, sweetened with natural sugars
like sucrose and corn syrup, and zero-calorie beverages include diet
soft drinks and teas, and water. Other beverages, such as juice and
milk, are excluded from the study due to challenges in identifying their
sugar contents. Nevertheless, the total sales information is kept in the
study to represent the overall patterns of beverage sales irrespective
of their sugar content.

\hypertarget{exploratory-analysis}{%
\subsection{3. Exploratory Analysis}\label{exploratory-analysis}}

Exploratory Data Analysis (EDA) is an essential precursor to formal
statistical methods, uncovering underlying patterns and outliers in the
data. It ensures the subsequent analysis is built on a solid foundation
of understanding, thereby enhancing the reliability of the findings.

\hypertarget{box-plots}{%
\subsubsection{3.1 Box Plots}\label{box-plots}}

The following boxplots represent the distribution of sugary and
zero-calorie beverage sales across different intervention strategies.
Each boxplot captures the sales data variability with the central line
denoting the median, the edges of the box indicating the interquartile
range (IQR), and the whiskers extending to the furthest points that are
not considered outliers. Outliers are individual points beyond the
whiskers. Note that due to the bimodality of sales, the boxplot
incorrectly indicates many outliers. The interventions, labelled on the
x-axis, include baseline (preint), discount \& messaging (dismes),
washout period (wash), the discount only (dis), second washout (wash2),
follow up (follow), calorie content poster (cal), the exercise required
to burn the calorie content (excer), and combination of the two (both).

\begin{figure}
\centering
\includegraphics{main_files/figure-latex/Boxplots-1.pdf}
\caption{The boxplots display the sales distribution of Sugary and
Zero-Calorie beverages across various interventions. Each intervention
is colour-coded and shows the range of sales data with the central line
representing the median sales.}
\end{figure}

\hypertarget{correlation-plot}{%
\subsubsection{3.2 Correlation Plot}\label{correlation-plot}}

The following plot investigates the correlation structure between the
day of the week (\texttt{DofW}), the number of zero-calorie drinks sold
(\texttt{ZeroCal}), and the number of sugary drinks sold
(\texttt{Sugary}). In this plot, the size, color of the circles and
number represent the strength of the correlation coefficients between
the variables. The \texttt{ZeroCal} and \texttt{Sugary} variables
exhibit a very strong positive correlation with a correlation
coefficient of 0.97. This suggests that sales of zero-calorie and sugary
drinks are closely related; when sales of one type increase, sales of
the other type tend to increase in a similar fashion. Conversely, both
\texttt{ZeroCal} and \texttt{Sugary} drinks show a negative correlation
with \texttt{DofW}, as indicated by the coefficient of -0.39. This
negative correlation suggests a tendency for the sales of both drink
types to decrease on certain days of the week.

\includegraphics{main_files/figure-latex/correlation-1.pdf}

\hypertarget{missing-values-and-data-imbalance}{%
\subsubsection{3.3 Missing Values and Data
Imbalance}\label{missing-values-and-data-imbalance}}

The data has some missing data. It is important to identify what kind of
missing data exists within a dataset to better understand how to handle
missingness during formal analysis. It appears that the missing data
qualifies as missing not at random (MNAR), meaning that the probability
of any given observation being missing varies for unidentified reasons.
It is also important to note from the observations counts whether or not
the data appears to be balanced since imbalanced data can hinder model
accuracy. Balance between sites appears to be reasonable, where some
imbalance is present between interventions. Namely, the `follow',
`wash', and `wash2' levels are imbalanced when compared to the other
interventions (see figure in appendix).

\hypertarget{time-series-plot}{%
\subsubsection{3.4 Time Series Plot}\label{time-series-plot}}

The following stacked line plot represents the sales time series of
zero-calorie and sugary beverages across different sites. Each line
represents the sales trajectory of one beverage type---green for
zero-calorie and blue for sugary drinks. The x-axis represents time (in
days), and the y-axis represents the sales volume. Dashed vertical lines
indicate the start of different interventions, labelled as dismes
(discount \& messaging), dis (only discount), cal (calorie content
poster), exer (exercise-based posters), and both. The interventions
appear to influence sales, as suggested by changes in the lines'
trajectories post-intervention. The plots are faceted by site, allowing
for a comparative view of sales patterns across different locations.

\begin{figure}
\centering
\includegraphics{main_files/figure-latex/sales-time-series-plot-1.pdf}
\caption{This plot illustrates the daily sales volumes of zero-calorie
(in green) and sugary (in blue) beverages across three hospital sites
over 30 weeks. The dashed and their corresponding shorthand labels mark
the interventions to allow for visual assessment of their impact on
beverage sales trends.}
\end{figure}

For more visual exploration see appendix.

\hypertarget{formal-analysis}{%
\subsection{4. Formal Analysis}\label{formal-analysis}}

Overall, three models are considered for formal analysis.

\hypertarget{iinterupted-times-series-analysis}{%
\subsubsection{4.1 Iinterupted Times Series
Analysis}\label{iinterupted-times-series-analysis}}

The study's design, an interrupted time series across multiple sites,
naturally lends itself to Interrupted Time Series Analysis (ITSA). This
method effectively handles the challenges posed by non-randomized
designs, isolating intervention impacts from pre-existing trends.

The ITSA can characterize the immediate and ongoing effects of visual
calorie content displays and price discounts on beverage selection. It
segments the data across different intervention phases to quantify
changes in zero-calorie and sugary beverage sales, offering a detailed
view of each strategy's effectiveness over time.

Moreover, ITSA's adaptability enables the analysis of variations in
intervention effects across sites, enabling site-specific nuances. This
approach reveals whether an intervention's success is uniform or
site-dependent but also compares the impact of combined interventions
against singular strategies. Therefore, employing ITSA can address the
study's core questions, demonstrating its suitability for unravelling
the effects of multifaceted interventions on consumer behaviour.

\hypertarget{generalized-estimating-equations}{%
\subsubsection{4.2 Generalized Estimating
Equations}\label{generalized-estimating-equations}}

The Generalized Estimating Equations (GEE) approach is a convenient and
relatively easy to interpret method to model longitudinal data. GEE is
suitable for analyzing the data from this study since daily sales of
bottled sugared beverages and zero-calorie beverages were measured
repeatedly over time. GEE can be thought of as an extension of the
Generalized Linear Model to longitudinal data (Columbia University).
This method is particularly convenient due to its high statistical
power, built-in handling of missing at random data, and its ability to
account for within-subject correlation in non-normal data.

Since the study aims to investigate the number of beverages sold at each
site, this method assumes an outcome of zero-calorie and sugary
beverages sold. Predictors include intervention type, site, day of the
week, and total beverage sales. Site, day of the week, and total
beverage sales predictors allow the model to adjust for any extraneous
effects and possible sale or time trends independent of the studies
interventions. Wash periods are excluded from the model and total sales
are used as a control instead. Since this method models count data, a
log link function is most appropriate, such as the Poisson or Negative
Binomial. Models may be fitted over all sites simultaneously, or as one
model per site. In the latter case, the sit factor may be excluded from
the model. It is appropriate to try both to examine comparable results.
Once the models are fitted, the GEE method will return coefficients for
every intervention or combination of interventions taken during the
study. These can be interpreted to help answer the studies main
objectives. Namely, to examine how each intervention affected
zero-calorie and sugary beverage sales, how sales differed by site, and
comparing the impacts between different interventions on zero-calorie
and sugary beverages sales. Hypothesis tests can be performed on each
coefficient to test intervention and site effects. A Bonferroni
correction is needed to adjust for increased risk of Type I error when
making multiple statistical tests.

\hypertarget{linear-mixed-effects-model}{%
\subsubsection{4.3 Linear Mixed Effects
Model}\label{linear-mixed-effects-model}}

Linear Mixed Effects (LME) models are useful for analyzing data
structured in clusters in a longitudinal study. Within a LME model,
fixed effects are those that are consistent across all observations,
such as the global influence of intervention and the day of week. These
effects are assumed to be the baseline of impact across all sites.
Random effects, on the other hand, account for differences between sites
or temporal fluctuations within a site that are not captured by the
fixed effects.

In this dataset, the five intervention methods across sites could be
transformed into four indicator variables, representing the presence and
absence of Discount, Additional Messaging, Calorie Display, Exercise
Display. The pre-intervension period and follow-up period are treated as
baseline reference observations. Variables that would be included in the
LME model would be day of week, four intervention indicators and
duration into the intervention, clustered by site. The model development
process involves selection of fixed effect and random effect parameters,
typically guided by statistical tests like the log-likelihood test.

Linear mixed effect models are used with the following assumptions:
error terms is required to be normally distributed with a consistent
spread, and the relationship between covariates and response are linear.
These assumptions can be verified with diagnostic plots such as
scatterplots and Q-Q plots of the errors.

\hypertarget{conclusions}{%
\subsection{5. Conclusions}\label{conclusions}}

The recommended statistical process for assessing the impact of
strategies to promote zero-calorie beverages over their sugary
alternatives involves both exploratory and formal analyses. An
exploratory data analysis will help identify underlying patterns in the
data, such as correlation and missingness. The formal analysis is
recommended to include three models: interrupted time series analysis,
generalized estimating equations, and the linear mixed-effects model.
Each of these models is capable of handling time-series and longitudinal
data. Results from these models can be tested and compared for security.
These analyses will answer if the data may indicate an impact on
beverage sales by various labeling and discount strategies.

\hypertarget{references}{%
\subsection{6. References}\label{references}}

Columbia University Mailman School of Public Health. (n.d.). Repeated
Measures Analysis. Columbia University Mailman School of Public Health.
\url{https://www.publichealth.columbia.edu/research/population-health-methods/repeated-measures-analysis}

\hypertarget{statistical-appendix}{%
\subsection{7. Statistical Appendix}\label{statistical-appendix}}

\begin{itemize}
\tightlist
\item
  Mathematical formulas.
\item
  Additional tables/figures.
\end{itemize}

\hypertarget{lme-model}{%
\subsubsection{LME model}\label{lme-model}}

Let \(y_i\) be the vector of measurements of response for site \(i\),
then a Linear Mixed Effect model is written as
\[y_i = X_i\beta + Z_i b_i + e_i\] where \(\beta\) contains fixed
effects, \(b_i\) contains random effects parameter for site \(i\). In
this case, the expanded general parameters would be
\[y_{ij}  = \beta_0 + b_{ij}^0 + (\beta_1 + b_{ij}^1)Treatment + (\beta_2 + b_{ij}^2)Day of Week +(\beta_3 + b_{ij}^3)Duration +e_{ij}\]
where \(y_{ij}\) is the sales of beverage at site \(i\) day \(j\), \$
\beta\^{}0 + b\_\{ij\}\^{}0\$ is the fixed and random terms for
intercept, \((\beta_1 + b_{ij}^1)Treatment\) consists of the fixed and
random terms for all four indicator variable for treatment,
\((\beta_2 + b_{ij}^2)Day of Week\) consists of the fixed and random
terms for each day of the week, \((\beta_3 + b_{ij}^3)Duration\)
represents the fixed and random effect of slope for duration into a
specific treatment.

\hypertarget{gee-model}{%
\subsubsection{GEE Model}\label{gee-model}}

Let \(\mathbf{Y_i}\) be the outcome variable for beverage \(i\)
(zero-calorie or sugared) sales.

Let \(g(\cdot)\) be the log link function (Poisson or Negative
Binomial). Then, for design matrix \(\mathbf{X}\) including all revenant
predictors, the model can be written more explicitly as

\[
g(\mathbb{E}[Y_i]) = \beta_{0i} \beta_{1i} (\text{Discount}) + \beta_{2i} (\text{Discount + Messaging}) + \beta_{3i} (\text{Calorie Messaging}) 
\]

\[
+ \beta_{4i} (\text{Exercise Messaging}) + \beta_{5i} (\text{Both Calorie Messaging}) + \beta_{6i} (\text{site B}) + \beta_{7i} (\text{site C})
\]

\[
+ \beta_{8i} (\text{Day of Week})+ \beta_{9i} (\text{Total sales})
\]

Then \(\beta_{0i}\) is the intercept. (Discount) and (Discount +
Messaging) are each dummy variables to represent the discount
intervention without messaging, and discount with messaging
respectively. (Site B) and (Site C) are also dummy variables to indicate
the site, with the baseline being site A.

\hypertarget{itsa-model}{%
\subsubsection{ITSA Model}\label{itsa-model}}

Interrupted Time Series Analysis (ITSA) with segmented regression is a
statistical technique tailored for quasi-experimental designs that
involve interventions at known time points. ITSA is particularly suited
for this study where interventions are sequentially introduced in a
multi-site setting and where the main interest lies in the impact on
sales of zero-calorie (ZeroCal) and sugary (Sugary) beverages.

The general form of the segmented regression model for ITSA applied to
this context can be expressed as:

\[Y_t = \beta_0 + \beta_1 T_t + \sum_{k=1}^{K} (\beta_{2k} I_{kt} + \beta_{3k} T_{kt} I_{kt}) + \epsilon_t \]

Where: - \(Y_t\) is the sales of beverages at time \(t\). - \(T_t\) is
the time since the start of the study (time trend). - \(I_{kt}\) is an
indicator for intervention \(k\) (0 before intervention k, 1 after
intervention k). - \(T_{kt}\) is the time since intervention \(k\)
started, multiplied by the intervention indicator. - \(\beta_0\) is the
intercept, representing the baseline level of sales. - \(\beta_1\) is
the coefficient for the time trend, representing the pre-intervention
trend of sales. - \(\beta_{2k}\) is the change in level immediately
after intervention \(k\). - \(\beta_{3k}\) is the change in trend after
intervention \(k\). - \(K\) is the total number of interventions. -
\(\epsilon_t\) is the error term which is assumed to be normally
distributed with mean zero and constant variance.

This model can be fitted separately for ZeroCal and Sugary sales to
ascertain the unique effects of interventions on each type of beverage.
The model can also be expanded to account for auto-correlated errors
which are common in time series data, by incorporating an AR(1) process
or other suitable autocorrelation structures.

For the investigation of site-specific effects, random effects or fixed
effects models can be used. A random effects model would be suitable if
we assume that the sites are a random sample from a larger population,
with the model taking the form:

\[Y_{it} = \beta_0 + \beta_1 T_t + u_i + \sum_{k=1}^{K} (\beta_{2k} I_{kt} + \beta_{3k} T_{kt} I_{kt}) + \epsilon_{it} \]

Where \(u_i\) is the random effect for site \(i\) and \(\epsilon_{it}\)
is the within-site error term.

By contrast, a fixed effects model would treat each site as a unique
entity and estimate site-specific intercepts:

\[Y_{it} = \beta_{0i} + \beta_1 T_t + \sum_{k=1}^{K} (\beta_{2k} I_{kt} + \beta_{3k} T_{kt} I_{kt}) + \epsilon_{it} \]

With \(\beta_{0i}\) being the intercept for site \(i\), allowing for
different baseline sales levels at each site.

The interaction terms \(\beta_{3k} T_{kt} I_{kt}\) are critical for
evaluating the sustained impact of interventions over time. If these
coefficients are significantly different from zero, it suggests that the
interventions had an effect beyond an immediate jump or drop in sales,
altering the underlying trend of beverage sales.

To evaluate the combined effect of interventions, interaction terms
between interventions can be included:

\[Y_{it} = \beta_0 + \beta_1 T_t + u_i + \sum_{k=1}^{K} \beta_{2k} I_{kt} + \sum_{k=1}^{K} \beta_{3k} T_{kt} I_{kt} + \sum_{k<l} \beta_{4kl} I_{kt} I_{lt} + \epsilon_{it} \]

Here, \(\beta_{4kl}\) captures the combined effect of interventions
\(k\) and \(l\) when both are in effect.

Lastly, the model can be augmented with covariates to control for other
factors that may influence sales, such as seasonal effects or marketing
campaigns. These covariates can be time-varying and should be included
in the model if they are thought to confound the relationship between
the interventions and sales.

\hypertarget{histogram-plots}{%
\subsubsection{Histogram Plots}\label{histogram-plots}}

The following histogram plots show the frequency distribution of sales
for Sugary (in purple), Zero-Calorie (ZeroCal, in teal), and Total (in
yellow) beverages. The x-axis of each histogram represents the sales
volume, while the y-axis indicates the count of observations within each
sales range. The pattern in all histograms is similar: most sales
numbers cluster at the lower end of the scale, suggesting a higher
frequency of days with fewer sales; however, the sales histograms
exhibit a second weaker mode, indicating two common sales volumes across
the observed period.

\begin{figure}
\centering
\includegraphics{main_files/figure-latex/Plot-Histograms-1.pdf}
\caption{Sales Distribution Analysis: Histograms displaying the
frequency of sales for Sugary (purple), Zero-Calorie (teal), and Total
combined (yellow) beverages. Each histogram reveals the distribution
pattern of sales volumes, highlighting the bimodal nature of sales
across all types.}
\end{figure}

\hypertarget{scatter-plot}{%
\subsubsection{Scatter Plot}\label{scatter-plot}}

The following scatter plot depicts the relationship between zero-calorie
and sugary beverage sales at three different hospital sites: A or chop
(purple), B or HF (blue), and C or NS (yellow). The x-axis represents
zero-calorie beverage sales, and the y-axis represents sugary beverage
sales. A dashed line, suggesting the line of equality, indicates where
the sales for both types would be equal. Points above the line indicate
higher sugary beverage sales when compared to zero-calorie ones, and
points below the line indicate the opposite. The clustering of points
towards the upper right suggests that for higher sales volumes, sugary
beverages tend to sell as much as or more than zero-calorie options,
particularly in site A (chop). The plot reveals variability in the sales
patterns across sites, with the HF site having a more direct correlation
between ZeroCal and Sugary sales when compared to other sites.

\begin{figure}
\centering
\includegraphics{main_files/figure-latex/Sales-Scatterplot-1.pdf}
\caption{This scatter plot contrasts zero-calorie and sugary beverage
sales, colour-coded by the site. Each point represents the paired sales
data for a given day, with the site-specific colour coding (chop in
purple, HF in blue, NS in yellow) illustrating the sales trend at each
location. The dashed diagonal line marks the parity where the sales of
both beverage types are equal. Deviations from this line highlight the
predominance of one beverage type over the other in daily sales.}
\end{figure}

\hypertarget{missing-values}{%
\subsubsection{Missing Values}\label{missing-values}}

The following plot visualizes the missing values.

\begin{figure}

{\centering \includegraphics[width=0.75\linewidth]{main_files/figure-latex/data quality-1} 

}

\caption{This plot provides insight into the frequency of missingness within the dataset. Black indicates missing data. Additionally it shows the quanitity of data available by site and by intervention.}\label{fig:data quality}
\end{figure}

\end{document}
