% Options for packages loaded elsewhere
\PassOptionsToPackage{unicode}{hyperref}
\PassOptionsToPackage{hyphens}{url}
%
\documentclass[
]{article}
\usepackage{amsmath,amssymb}
\usepackage{iftex}
\ifPDFTeX
  \usepackage[T1]{fontenc}
  \usepackage[utf8]{inputenc}
  \usepackage{textcomp} % provide euro and other symbols
\else % if luatex or xetex
  \usepackage{unicode-math} % this also loads fontspec
  \defaultfontfeatures{Scale=MatchLowercase}
  \defaultfontfeatures[\rmfamily]{Ligatures=TeX,Scale=1}
\fi
\usepackage{lmodern}
\ifPDFTeX\else
  % xetex/luatex font selection
\fi
% Use upquote if available, for straight quotes in verbatim environments
\IfFileExists{upquote.sty}{\usepackage{upquote}}{}
\IfFileExists{microtype.sty}{% use microtype if available
  \usepackage[]{microtype}
  \UseMicrotypeSet[protrusion]{basicmath} % disable protrusion for tt fonts
}{}
\makeatletter
\@ifundefined{KOMAClassName}{% if non-KOMA class
  \IfFileExists{parskip.sty}{%
    \usepackage{parskip}
  }{% else
    \setlength{\parindent}{0pt}
    \setlength{\parskip}{6pt plus 2pt minus 1pt}}
}{% if KOMA class
  \KOMAoptions{parskip=half}}
\makeatother
\usepackage{xcolor}
\usepackage[margin=1in]{geometry}
\usepackage{graphicx}
\makeatletter
\def\maxwidth{\ifdim\Gin@nat@width>\linewidth\linewidth\else\Gin@nat@width\fi}
\def\maxheight{\ifdim\Gin@nat@height>\textheight\textheight\else\Gin@nat@height\fi}
\makeatother
% Scale images if necessary, so that they will not overflow the page
% margins by default, and it is still possible to overwrite the defaults
% using explicit options in \includegraphics[width, height, ...]{}
\setkeys{Gin}{width=\maxwidth,height=\maxheight,keepaspectratio}
% Set default figure placement to htbp
\makeatletter
\def\fps@figure{htbp}
\makeatother
\setlength{\emergencystretch}{3em} % prevent overfull lines
\providecommand{\tightlist}{%
  \setlength{\itemsep}{0pt}\setlength{\parskip}{0pt}}
\setcounter{secnumdepth}{-\maxdimen} % remove section numbering
\usepackage{fontspec}
\usepackage{caption}
\captionsetup{font={small, it}, labelfont=bf}
\setmainfont{Times New Roman}
\fontsize{12pt}{15pt}\selectfont
\usepackage{xcolor}
\usepackage{setspace}
\setstretch{1.2}
\ifLuaTeX
  \usepackage{selnolig}  % disable illegal ligatures
\fi
\IfFileExists{bookmark.sty}{\usepackage{bookmark}}{\usepackage{hyperref}}
\IfFileExists{xurl.sty}{\usepackage{xurl}}{} % add URL line breaks if available
\urlstyle{same}
\hypersetup{
  pdftitle={Statistical Advice on the Effect of Interventions on Beverage Sales},
  hidelinks,
  pdfcreator={LaTeX via pandoc}}

\title{Statistical Advice on the Effect of Interventions on Beverage
Sales}
\author{Parham Pishrobat (71097927),\\
Johnson Chen (85784080),\\
Sarah Masri (97415681)}
\date{March 28, 2024}

\begin{document}
\maketitle

\pagebreak

\hypertarget{introduction}{%
\subsection{1. Introduction}\label{introduction}}

Concerns around sugar consumption and its health implications have
prompted many interventions to change consumer behaviour cut down on
sugary beverages. The current study investigates the effectiveness of
two types of intervention strategies to motivate and incentivize
consumers to choose zero-calorie beverages over sugary alternatives. In
particular, the research question focuses on the impact of two
strategies to inform consumers about calorie content through visual
presentations: posters highlighting either the calorie content or the
physical activity required to burn calories. Furthermore, the
effectiveness of price discounts on behaviour is explored, both
independently and in conjunction with explanatory messaging. This study
also seek to understand if the effectiveness of those strategies varies
across different sites.

\hypertarget{data-description-and-summaries}{%
\subsection{2. Data Description and
Summaries}\label{data-description-and-summaries}}

The study adopts an interrupted time-series multi-site
quasi-experimental design to assess the effectiveness of the five
interventions on the purchase patterns of bottled sugary and
zero-calorie beverages. The data are recorded from cafeterias and
convenience shops at three hospital sites, denoted by A, B, and C.
Hospital A is urban and has two cafeterias and two convenience shops.
Hospital B is also an urban setting but it has only one cafeteria.
Hospital C is a suburban setting, having one cafeteria and one
convenience shop. Both interventions and data collection were automatic
at site A and by trained personnel in sites B and C. Sugary beverages
include regular soft drinks and iced teas, sweetened with natural sugars
like sucrose and corn syrup, and zero-calorie beverages include diet
soft drinks and teas, and water. Other beverages, such as juice and
milk, are excluded from the study due to challenges in identifying their
sugar contents. Nevertheless, the total sales information is kept in the
study to represent the overall patterns of beverage sales irrespective
of their sugar content.

The experiment, starting on October 27, 2009 and ending 32 weeks later,
measure daily sales of bottled sugary and zero-calorie beverages. The
study period included a baseline data collection phase, intervention
phases to elicit behaviour change, and washout periods to assess the
persistence of intervention effects. The dataset contains 631 sales
counts for zero-calorie, sugary, and all drinks, sale location, day of
the week, and the type of intervention applied. The recorded variables
consist of sales count, day of the week, site location, type of
intervention, and beverage category (zero-calorie and sugary options).
The price interventions consist of two periods of 10\% discount on
zero-calorie beverages, with one phase providing additional explanatory
messaging about the discount. The calorie messaging interventions
provided information on the caloric content of sugary drinks, the
physical activity required to burn off these calories, and a combination
of both strategies.

The day of week, site and intervention covariates are each considered
categorical data types. Other observations are classified under
numerical data types as they measure sales counts. Missing data is
observed over some control periods of the study.

\hypertarget{exploratory-data-analysis}{%
\subsection{3. Exploratory Data
Analysis}\label{exploratory-data-analysis}}

Exploratory Data Analysis (EDA) is an important first step in data
analysis in uncovering underlying patterns, relationships, and outliers
in the data. It ensures the subsequent analysis is built on a solid
understanding of the data, thereby enhancing the reliability of the
findings. To explore the data of the study, box plots and time series
plots are highly important and included in the main report. The
visualization in particular provide insight into the distribution of
sugary and zero-calorie beverage sales, as well as their temporal
trends. Additionally, Appendix A provides supplementary visualizations
encompassing missing values, histograms and a correlation plot.
Collectively, these exploratory techniques build a foundation and
justifies the development of our formal analysis, aiming to explore the
impact product labeling on beverage sales.

\hypertarget{boxplot}{%
\subsubsection{3.1 Boxplot}\label{boxplot}}

A boxplot is a method for graphically demonstrating the property of
statistical distribution of numerical data. This following boxplot shows
the distribution of sugary and zero-calorie beverage sales across
different intervention strategies. Each boxplot captures the sales data
variability with the central line denoting the median, the edges of the
box indicating the interquartile range (IQR), and the whiskers extending
to the furthest points that are not considered outliers. Outliers are
individual points beyond the whiskers.

Note that due to the bimodality of sales, the boxplot incorrectly
indicates many outliers. The interventions, labelled on the x-axis,
include baseline (preint), discount \& messaging (dismes), washout
period (wash), the discount only (dis), second washout (wash2), follow
up (follow), calorie content poster (cal), the exercise required to burn
the calorie content (excer), and combination of the two (both).

\includegraphics{main_files/figure-latex/Boxplots-1.pdf}
\includegraphics{main_files/figure-latex/Boxplots-2.pdf}

\hypertarget{correlation-plot}{%
\subsubsection{3.2 Correlation Plot}\label{correlation-plot}}

The following plot investigates the correlation structure between the
day of the week (\texttt{DofW}), the number of zero-calorie drinks sold
(\texttt{ZeroCal}), and the number of sugary drinks sold
(\texttt{Sugary}). In this plot, the size, color of the circles and
number represent the strength of the correlation coefficients between
the variables. The \texttt{ZeroCal} and \texttt{Sugary} variables
exhibit a very strong positive correlation with a correlation
coefficient of 0.97. This suggests that sales of zero-calorie and sugary
drinks are closely related; when sales of one type increase, sales of
the other type tend to increase in a similar fashion. Conversely, both
\texttt{ZeroCal} and \texttt{Sugary} drinks show a negative correlation
with \texttt{DofW}, as indicated by the coefficient of -0.39. This
negative correlation suggests a tendency for the sales of both drink
types to decrease on certain days of the week.

\hypertarget{time-series-plot}{%
\subsubsection{3.3 Time Series Plot}\label{time-series-plot}}

The following stacked line plot represents the sales time series of
zero-calorie and sugary beverages across different sites. Each line
represents the sales trajectory of one beverage type---green for
zero-calorie and blue for sugary drinks. The x-axis represents time (in
days), and the y-axis represents the sales volume. Dashed vertical lines
indicate the start of different interventions, labelled as dismes
(discount \& messaging), dis (only discount), cal (calorie content
poster), exer (exercise-based posters), and both. The interventions
appear to influence sales, as suggested by changes in the lines'
trajectories post-intervention. The plots are faceted by site, allowing
for a comparative view of sales patterns across different locations.

\begin{figure}
\centering
\includegraphics{main_files/figure-latex/sales-time-series-plot-1.pdf}
\caption{This plot illustrates the daily sales volumes of zero-calorie
(in green) and sugary (in blue) beverages across three hospital sites
over 30 weeks. The dashed and their corresponding shorthand labels mark
the interventions to allow for visual assessment of their impact on
beverage sales trends.}
\end{figure}

\hypertarget{missing-values-and-data-imbalance}{%
\subsubsection{3.3 Missing Values and Data
Imbalance}\label{missing-values-and-data-imbalance}}

The dataset contains some missing values. It is important to identify
what kind of missing data exists within a dataset to better understand
how to handle missingness during formal analysis. There are 9 days of
unrecorded zero-calorie and sugary beverage sales, 7 of which represent
the last week of the study at site B. Failing to address missing data
may lead to a reduction of statistical power and biased results. It
appears that the missing data are missing not at random (MNAR), meaning
that the probability of any given observation being missing varies for
unidentified reasons (e.g., hospital closures, public events). This
supports the appropriateness in choice of model in the formal analysis.

It is also important to note whether of not the dataset reflects a
roughly equal number of observations between the sites and interventions
respectively. If one class in either the site or interventions are
represented disproportionately, then models used may become biased
towards the most frequently seen class. Balance between sites appears to
be reasonable, where some imbalance is present between interventions.
Namely, the `follow', `wash', and `wash2' levels are imbalanced when
compared to the other interventions.

\hypertarget{formal-analysis}{%
\subsection{4. Formal Analysis}\label{formal-analysis}}

Informed by the exploratory analysis above, three models are suggested
for the formal analysis: interrupted times series, generalized
estimating equations, and a linear mixed effects model.

\hypertarget{interrupted-times-series-analysis}{%
\subsubsection{4.1 Interrupted Times Series
Analysis}\label{interrupted-times-series-analysis}}

The study's design, an interrupted time series across multiple sites,
naturally lends itself to Interrupted Time Series Analysis (ITSA). This
method effectively handles the challenges posed by non-randomized
designs, isolating intervention impacts from pre-existing trends.

The ITSA method can characterize the immediate and ongoing effects of
visual calorie content displays and price discounts on beverage
selection. It segments the data across different intervention phases to
quantify changes in zero-calorie and sugary beverage sales, offering a
detailed view of each strategy's effectiveness over time.

Moreover, ITSA's adaptability enables the analysis of variations in
intervention effects across sites, enabling site-specific nuances. This
approach reveals whether an intervention's success is uniform or
site-dependent by comparing the impact of combined interventions with
that of singular strategies. Therefore, employing ITSA can address the
study's core questions, demonstrating its suitability for unravelling
the effects of multifaceted interventions on consumer behaviour.

\begin{verbatim}
## Loading required package: Matrix
\end{verbatim}

\begin{verbatim}
## 
## Attaching package: 'Matrix'
\end{verbatim}

\begin{verbatim}
## The following objects are masked from 'package:tidyr':
## 
##     expand, pack, unpack
\end{verbatim}

\begin{verbatim}
## 
## Attaching package: 'lmerTest'
\end{verbatim}

\begin{verbatim}
## The following object is masked from 'package:lme4':
## 
##     lmer
\end{verbatim}

\begin{verbatim}
## The following object is masked from 'package:stats':
## 
##     step
\end{verbatim}

\hypertarget{generalized-estimating-equations}{%
\subsubsection{4.2 Generalized Estimating
Equations}\label{generalized-estimating-equations}}

An alternative approach might be forming The Generalized Estimating
Equations (GEE) model. GEE is a convenient and relatively easy to
interpret method to model longitudinal data. GEE is suitable for
analyzing the data of from this study since daily sales of bottled
sugared beverages and zero-calorie beverages are measured repeatedly
over time. GEE can be thought of as an extension of the Generalized
Linear Model to longitudinal data (Columbia University). This method is
particularly convenient due to its high statistical power, built-in
handling of missing at random data, and its ability to account for
within-subject correlation in non-normal data.

Since the study aims to investigate the amount of beverages sold at each
site, this method assumes an outcome of zero-calorie and sugary
beverages sold. Predictors include intervention type, site, day of the
week, and total beverage sales. Site, day of the week, and total
beverage sales predictors allow the model to adjust for any extraneous
effects and possible sale or time trends independent of the studies
interventions. Wash periods are excluded from the model and total sales
are used as a control instead. Since this method models count data, a
log link function is most appropriate, such as the Poisson or Negative
Binomial. Models may be fitted over all sites simultaneously, or as one
model per site. In the latter case, the sit factor may be excluded from
the model. It is appropriate to try both to examine comparable results.
Once the models are fitted, the GEE method will return coefficients for
every intervention or combination of interventions taken during the
study. These can be interpreted to help answer the studies main
objectives. Namely, to examine how each intervention affects
zero-calorie and sugary beverage sales, how sales differed by site, and
comparing the impacts between different interventions on zero-calorie
and sugary beverages sales. Hypothesis tests can be performed on each
coefficient to test intervention and site effects. A Bonferroni
correction is needed to adjust for increased risk of Type I error when
making multiple statistical tests.

\hypertarget{linear-mixed-effects-model}{%
\subsubsection{4.3 Linear Mixed Effects
Model}\label{linear-mixed-effects-model}}

Linear Mixed Effects (LME) models are useful for analyzing data
structured in clusters in a longitudinal study. Within an LME model,
fixed effects are those that are consistent across all observations,
such as the global influence of intervention and the day of week. These
effects are assumed to be the baseline of impact across all sites.
Random effects, on the other hand, account for differences between sites
or temporal fluctuations within a site that are not captured by the
fixed effects.

In this dataset, the five intervention methods across sites could be
transformed into four indicator variables, representing the presence and
absence of Discount, Additional Messaging, Calorie Display, and Exercise
Display. The pre-intervension period and follow-up period are treated as
baseline reference observations. Variables that would be included in the
LME model would be day of week, four intervention indicators and
duration into the intervention, clustered by site. The model development
process involves selection of fixed effect and random effect parameters,
typically guided by statistical tests like the log-likelihood test.

\hypertarget{conclusions}{%
\subsection{5. Conclusions}\label{conclusions}}

Both exploratory and formal analyses are recommended for investigating
the effectiveness of intervention strategies to promote zero-calorie
beverages over their sugary alternatives involves both exploratory and
formal analyses. An exploratory data analysis will help identify
underlying patterns in the data, such as correlation and missingness.
The formal analysis is recommended to includes three models: interrupted
time series analysis, generalized estimating equations, and the linear
mixed-effects model. Each of these models is capable of handling
time-series and longitudinal data. Results from these models can be
tested and compared for security. These analyses will answer if the data
may indicate an impact on beverage sales by various labeling and
discount strategies.

\pagebreak

\hypertarget{references}{%
\subsection{6. References}\label{references}}

Columbia University Mailman School of Public Health. (n.d.). Repeated
Measures Analysis. Columbia University Mailman School of Public Health.
\url{https://www.publichealth.columbia.edu/research/population-health-methods/repeated-measures-analysis}

UCLA Statistical Consulting Group. (n.d.). Introduction to Linear Mixed
Models. Retrieved March 1, 2024, from
\url{https://stats.oarc.ucla.edu/other/mult-pkg/introduction-to-linear-mixed-models/}

University of Virginia Library. (n.d.). Getting Started with Generalized
Estimating Equations. Retrieved March 1, 2024, from
\url{https://library.virginia.edu/data/articles/getting-started-with-generalized-estimating-equations}

\hypertarget{other-resources}{%
\subsubsection{Other Resources:}\label{other-resources}}

\begin{itemize}
\tightlist
\item
  \href{https://rpubs.com/mbounthavong/MEPS_tutorial_6_itsa}{\underline{\textcolor{blue}{Interrupted Time Series Analysis}}}
\item
  \href{https://library.virginia.edu/data/articles/getting-started-with-generalized-estimating-equations\#:~:text=Generalized\%20estimating\%20equations\%2C\%20or\%20GEE,(i.e.\%2C\%20model\%20coefficients).}{\underline{\textcolor{blue}{Getting Started with Generalized Estimating Equations}}}
\item
  Linear Mixed Effect Model Tutorials:

  \begin{itemize}
  \tightlist
  \item
    \href{https://jontalle.web.engr.illinois.edu/MISC/lme4/bw_LME_tutorial.pdf}{\underline{\textcolor{blue}{LMEM tutorial (illinois.edu)}}}
  \item
    \href{https://ademos.people.uic.edu/Chapter17.html}{\underline{\textcolor{blue}{Chapter 17: Mixed Effects Modeling (uic.edu)}}}
  \item
    \href{https://stats.idre.ucla.edu/other/mult-pkg/introduction-to-linear-mixed-models/}{\underline{\textcolor{blue}{Introduction to Linear Mixed Models (ucla.edu)}}}
  \item
    \href{https://bodo-winter.net/tutorial/bw_LME_tutorial1.pdf}{\underline{\textcolor{blue}{Mixed Effects Models (bodywinter.com)}}}
  \item
    \href{https://ourcodingclub.github.io/tutorials/mixed-models/}{\underline{\textcolor{blue}{Mixed Effects, lme4 Tutorial}}}
  \item
    \href{https://cran.r-project.org/web/packages/lme4/vignettes/lmer.pdf}{\underline{\textcolor{blue}{Fitting Linear Mixed-Effects Models Using lme4}}}
  \end{itemize}
\item
  ASDA Teaching Recordings:

  \begin{itemize}
  \tightlist
  \item
    \href{https://community.grad.ubc.ca/gps/event/22318}{\underline{\textcolor{blue}{Exploratory Data Analysis}}}
  \item
    \href{https://community.grad.ubc.ca/gps/event/22327}{\underline{\textcolor{blue}{Study Design and Data Collection Essentials}}}
  \item
    \href{https://community.grad.ubc.ca/gps/event/22330}{\underline{\textcolor{blue}{Mixed Effects Models}}}
  \end{itemize}
\item
  Multiple Testing Correction Techniques:

  \begin{itemize}
  \tightlist
  \item
    \href{https://www.stat.berkeley.edu/~mgoldman/Section0402.pdf}{\underline{\textcolor{blue}{Multiple Testing Correction Techniques}}}
  \item
    \href{https://egap.org/resource/10-things-to-know-about-multiple-comparisons/}{\underline{\textcolor{blue}{An Article on the Multiple Testing Problem}}}
  \end{itemize}
\item
  Model Assumptions and Interpretations:

  \begin{itemize}
  \tightlist
  \item
    \href{https://besjournals.onlinelibrary.wiley.com/doi/10.1111/2041-210X.13434}{\underline{\textcolor{blue}{Linear Mixed Effects Model Assumptions}}}
  \item
    \href{https://cran.r-project.org/web/packages/DHARMa/vignettes/DHARMa.html}{\underline{\textcolor{blue}{DHARMa: a Residual Diagnostics Tool for LMEM}}}
  \item
    \href{www.medicine.mcgill.ca/epidemiology/Joseph/courses/EPIB-621/interaction.pdf}{\underline{\textcolor{blue}{Interactions Explained}}}
  \item
    \href{https://www.theanalysisfactor.com/interactions-categorical-and-continuous-variables/}{\underline{\textcolor{blue}{Understanding Interactions between Categorical and Continuous Variables}}}
  \item
    \href{https://ademos.people.uic.edu/Chapter13.html}{\underline{\textcolor{blue}{Interaction Plots}}}
  \item
    \href{https://rpubs.com/tf_peterson/interactionplotDemo}{\underline{\textcolor{blue}{Interaction Plots}}}
  \item
    \href{https://mcfromnz.wordpress.com/2011/03/02/anova-type-iiiiii-ss-explained/}{\underline{\textcolor{blue}{Testing for Interaction Significance}}}
  \item
    \href{https://cran.r-project.org/web/packages/emmeans/vignettes/interactions.html}{\underline{\textcolor{blue}{Package `emmeans` for Post-hoc Analysis and Pairwise Comparisons of the Interaction Effect}}}
  \item
    \href{https://stats.oarc.ucla.edu/r/seminars/interactions-r/}{\underline{\textcolor{blue}{Interaction Analysis in R}}}
  \item
    \href{https://cran.r-project.org/web/packages/emmeans/vignettes/comparisons.html}{\underline{\textcolor{blue}{Comparisons and Contrasts in emmeans}}}
  \end{itemize}
\end{itemize}

\pagebreak

\hypertarget{appendix-a-figures}{%
\subsection{Appendix A: Figures}\label{appendix-a-figures}}

The following sub section contain some additional information and
analysis results.

\hypertarget{histogram-plots}{%
\subsubsection{Histogram Plots}\label{histogram-plots}}

The following histogram plots show the frequency distribution of sales
for Sugary (in purple), Zero-Calorie (ZeroCal, in teal), and Total (in
yellow) beverages. The x-axis of each histogram represents the sales
volume, while the y-axis indicates the count of observations within each
sales range. The pattern in all histograms is similar: most sales
numbers cluster at the lower end of the scale, suggesting a higher
frequency of days with fewer sales; however, the sales histograms
exhibit a second weaker mode, indicating two common sales volumes across
the observed period.

\begin{figure}
\centering
\includegraphics{main_files/figure-latex/Plot-Histograms-1.pdf}
\caption{Sales Distribution Analysis: Histograms displaying the
frequency of sales for Sugary (purple), Zero-Calorie (teal), and Total
combined (yellow) beverages. Each histogram reveals the distribution
pattern of sales volumes, highlighting the bimodal nature of sales
across all types.}
\end{figure}

\hypertarget{scatter-plot}{%
\subsubsection{Scatter Plot}\label{scatter-plot}}

The following scatter plot depicts the relationship between zero-calorie
and sugary beverage sales at three different hospital sites: A or chop
(purple), B or HF (blue), and C or NS (yellow). The x-axis represents
zero-calorie beverage sales, and the y-axis represents sugary beverage
sales. A dashed line, suggesting the line of equality, indicates where
the sales for both types would be equal. Points above the line indicate
higher sugary beverage sales when compared to zero-calorie ones, and
points below the line indicate the opposite. The clustering of points
towards the upper right suggests that for higher sales volumes, sugary
beverages tend to sell as much as or more than zero-calorie options,
particularly in site A (chop). The plot reveals variability in the sales
patterns across sites, with the HF site having a more direct correlation
between ZeroCal and Sugary sales when compared to other sites.

\begin{figure}
\centering
\includegraphics{main_files/figure-latex/Sales-Scatterplot-1.pdf}
\caption{This scatter plot contrasts zero-calorie and sugary beverage
sales, colour-coded by the site. Each point represents the paired sales
data for a given day, with the site-specific colour coding (chop in
purple, HF in blue, NS in yellow) illustrating the sales trend at each
location. The dashed diagonal line marks the parity where the sales of
both beverage types are equal. Deviations from this line highlight the
predominance of one beverage type over the other in daily sales.}
\end{figure}

\hypertarget{correlation-plot-1}{%
\subsubsection{Correlation Plot}\label{correlation-plot-1}}

The following plot visualizes the correlation between numeric variables.

\begin{center}\includegraphics{main_files/figure-latex/correlation-1} \end{center}

\hypertarget{missing-values}{%
\subsubsection{Missing Values}\label{missing-values}}

The following plot visualizes the missing values.

\begin{figure}

{\centering \includegraphics[width=0.75\linewidth]{main_files/figure-latex/data-quality-1} 

}

\caption{This plot provides insight into the frequency of missingness within the dataset. Black indicates missing data. Additionally it shows the quanitity of data available by site and by intervention.}\label{fig:data-quality}
\end{figure}

\pagebreak

\hypertarget{appendix-b-models}{%
\subsection{Appendix B: Models}\label{appendix-b-models}}

\hypertarget{itsa-model}{%
\subsubsection{ITSA Model}\label{itsa-model}}

Interrupted Time Series Analysis (ITSA) with segmented regression is a
statistical technique tailored for quasi-experimental designs that
involve interventions at known time points. ITSA is particularly suited
for this study where interventions are sequentially introduced in a
multi-site setting and where the main interest lies in the impact on
sales of zero-calorie (ZeroCal) and sugary (Sugary) beverages.

The general form of the segmented regression model for ITSA applied to
this context can be expressed as:

\[Y_t = \beta_0 + \beta_1 T_t + \sum_{k=1}^{K} (\beta_{2k} I_{kt} + \beta_{3k} T_{kt} I_{kt}) + \epsilon_t \]

Where: - \(Y_t\) is the sales of beverages at time \(t\). - \(T_t\) is
the time since the start of the study (time trend). - \(I_{kt}\) is an
indicator for intervention \(k\) (0 before intervention k, 1 after
intervention k). - \(T_{kt}\) is the time since intervention \(k\)
started, multiplied by the intervention indicator. - \(\beta_0\) is the
intercept, representing the baseline level of sales. - \(\beta_1\) is
the coefficient for the time trend, representing the pre-intervention
trend of sales. - \(\beta_{2k}\) is the change in level immediately
after intervention \(k\). - \(\beta_{3k}\) is the change in trend after
intervention \(k\). - \(K\) is the total number of interventions. -
\(\epsilon_t\) is the error term which is assumed to be normally
distributed with mean zero and constant variance.

This model can be fitted separately for ZeroCal and Sugary sales to
ascertain the unique effects of interventions on each type of beverage.
The model can also be expanded to account for auto-correlated errors
which are common in time series data, by incorporating an AR(1) process
or other suitable autocorrelation structures.

For the investigation of site-specific effects, random effects or fixed
effects models can be used. A random effects model would be suitable if
we assume that the sites are a random sample from a larger population,
with the model taking the form:

\[Y_{it} = \beta_0 + \beta_1 T_t + u_i + \sum_{k=1}^{K} (\beta_{2k} I_{kt} + \beta_{3k} T_{kt} I_{kt}) + \epsilon_{it} \]

Where \(u_i\) is the random effect for site \(i\) and \(\epsilon_{it}\)
is the within-site error term.

By contrast, a fixed effects model would treat each site as a unique
entity and estimate site-specific intercepts:

\[Y_{it} = \beta_{0i} + \beta_1 T_t + \sum_{k=1}^{K} (\beta_{2k} I_{kt} + \beta_{3k} T_{kt} I_{kt}) + \epsilon_{it} \]

With \(\beta_{0i}\) being the intercept for site \(i\), allowing for
different baseline sales levels at each site.

The interaction terms \(\beta_{3k} T_{kt} I_{kt}\) are critical for
evaluating the sustained impact of interventions over time. If these
coefficients are significantly different from zero, it suggests that the
interventions had an effect beyond an immediate jump or drop in sales,
altering the underlying trend of beverage sales.

To evaluate the combined effect of interventions, interaction terms
between interventions can be included:

\[Y_{it} = \beta_0 + \beta_1 T_t + u_i + \sum_{k=1}^{K} \beta_{2k} I_{kt} + \sum_{k=1}^{K} \beta_{3k} T_{kt} I_{kt} + \sum_{k<l} \beta_{4kl} I_{kt} I_{lt} + \epsilon_{it} \]

Here, \(\beta_{4kl}\) captures the combined effect of interventions
\(k\) and \(l\) when both are in effect.

Lastly, the model can be augmented with covariates to control for other
factors that may influence sales, such as seasonal effects or marketing
campaigns. These covariates can be time-varying and should be included
in the model if they are thought to confound the relationship between
the interventions and sales.

\hypertarget{gee-model}{%
\subsubsection{GEE Model}\label{gee-model}}

Let \(\mathbf{Y_i}\) be the outcome variable for beverage \(i\)
(zero-calorie or sugared) sales.

Let \(g(\cdot)\) be the log link function (Poisson or Negative
Binomial). Then, for design matrix \(\mathbf{X}\) including all revenant
predictors, the model can be written more explicitly as

\[
g(\mathbb{E}[Y_i]) = \beta_{0i} + \beta_{1i} (\text{Discount}) + \beta_{2i} (\text{Discount + Messaging}) + \beta_{3i} (\text{Calorie Messaging}) 
\]

\[
+ \beta_{4i} (\text{Exercise Messaging}) + \beta_{5i} (\text{Both Calorie Messaging}) + \beta_{6i} (\text{site B}) + \beta_{7i} (\text{site C})
\]

\[
+ \beta_{8i} (\text{Day of Week})+ \beta_{9i} (\text{Total sales})
\]

Then \(\beta_{0i}\) is the intercept. (Discount) and (Discount +
Messaging) are each dummy variables to represent the discount
intervention without messaging, and discount with messaging
respectively. (Site B) and (Site C) are also dummy variables to indicate
the site, with the baseline being site A.

\hypertarget{lme-model}{%
\subsubsection{LME model}\label{lme-model}}

Let \(y_i\) be the vector of measurements of response for site \(i\),
then a Linear Mixed Effect model is written as
\[y_i = X_i\beta + Z_i b_i + e_i\] where \(\beta\) contains fixed
effects, \(b_i\) contains random effects parameter for site \(i\). In
this case, the expanded general parameters would be
\[y_{ij}  = \beta_0 + b_{ij}^0 + (\beta_1 + b_{ij}^1)Treatment + (\beta_2 + b_{ij}^2)Day of Week +(\beta_3 + b_{ij}^3)Duration +e_{ij}\]
where \(y_{ij}\) is the sales of beverage at site \(i\) day \(j\), \$
\beta\^{}0 + b\_\{ij\}\^{}0\$ is the fixed and random terms for
intercept, \((\beta_1 + b_{ij}^1)Treatment\) consists of the fixed and
random terms for all four indicator variable for treatment,
\((\beta_2 + b_{ij}^2)Day of Week\) consists of the fixed and random
terms for each day of the week, \((\beta_3 + b_{ij}^3)Duration\)
represents the fixed and random effect of slope for duration into a
specific treatment.

\pagebreak

\hypertarget{contributions}{%
\subsection{Contributions}\label{contributions}}

\textbf{Parham Pishrobat (71097927):} Introduction, Data, ITSA, Other EDA Plots, Formatting\\ \\ \\
\textbf{Johnson Chen (85784080):}  \quad LMEM, Correlation, Appendix, Formatting\\ \\ \\
\textbf{Sarah Masri (97415681):} \quad \quad GEE, Missing Values, Conclusion, Formatting\\ \\ \\

\end{document}
